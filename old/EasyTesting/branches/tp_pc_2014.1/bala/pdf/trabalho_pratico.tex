\documentclass{article}
\usepackage[utf8]{inputenc}

\title{Trabalho Prático 1}
%\author{feliperviegas }
\date{April 2014}

\usepackage{natbib}
\usepackage{graphicx}

\begin{document}

\maketitle

\section{Descrição do Problema}
    Neste trabalho aplicaremos os conceitos de Programação de Computadores para implementar um jogo baseado em laçamento de projéteis. O jogo funciona da seguinte maneira, dois jogadores são desafiados a lançar objetos uns aos outros e aquele que acertar o seu oponente primeiro vence. 
    
    Para descobrir se o jogador conseguiu acertar o seu oponente vamos utilizar a fórmula~\ref{eq} de lançamento descrita abaixo. Os objetos sempre são lançados a partir da posição do jogador. 
    
    Os dois jogadores nunca jogam simultaneamente, cada jogador tem a sua chance para acertar o seu oponente. Para lançar o objeto, o jogador deve dizer a velocidade inicial do lançamento e o ângulo de inclinação. Se o jogador acertar o chão, ele passa a vez, o segundo jogador tem a sua chance e assim por diante. O jogo só termina quando um jogador acertar o adversário. 

\section{Lançamento Inclinado}
    O lançamento inclinado estuda o movimento de um corpo no vácuo, cuja trajetória de lançamento forma um ângulo em relação à horizontal, como podemos observar na Figura~\ref{fig1}. Neste trabalho, estamos interessados em descobrir qual a distância máxima percorrida pelo objeto, dado a sua velocidade inicial e o ângulo de lançamento, fórmula~\ref{eq}.

\begin{center}
\includegraphics[scale=0.6]{img/fig.png}
\label{fig1}
\end{center}
    
\begin{equation}
d\_{max} = \frac{v_o^2 \times sen(2\times \theta)}{g}
\label{eq}
\end{equation}

Onde $v_o$ é a velocidade inicial do objeto no tempo zero, $\theta$ é o ângulo de lançamento e $g$ é a aceleração da gravidade, vamos considerar a aceleração ao nível do mar ($g = 9.80665 m/s^2$).



\section{Etapas do Trabalho}
Você deve ler de um arquivo de texto contendo o cenário inicial do jogo. Este arquivo deve possuir:
\begin{itemize}
    \item 100 caracteres.
    \item Caracteres permitidos são letras (maiúsculas ou minúsculas), para representar os jogadores e $\_$ para representar o chão.
    \item Os dois jogadores podem estar organizados de qualquer forma, não existe restrição para a posição dos jogadores.
\end{itemize}
    %40 caracteres
    Exemplo de entrada:
    
    \begin{center}
    $\_\_\_\_J\_\_\_\_\_\_\_\_\_\_\_\_\_\_\_\_\_\_\_\_\_\_\_\_\_\_\_\_\_\_\_\_\_M\_$
    \end{center}

    Você deve ler o arquivo de entrada para carregar o cenário inicial. Cada caractere representa um intervalo de distância igual a $10$ metros. No exemplo acima, a distância máxima do cenário é de $400$ metros ($40$ caracteres), onde o primeiro caractere representa os $10$ primeiros metros (intervalo de $0$ a $10$ metros), o segundo caractere representa o intervalo de $10$ a $20$ metros, e assim por diante. No cenário com $100$ caracteres, a distância máxima permitida é de $1000$ metros.
    
    A cada jogada, o jogador deve digitar a velocidade inicial e o ângulo de inclinação do objeto e usar a fórmula de lançamento para calcular a distância máxima percorrida pelo objeto. Você vai usar esta distância para descobrir em qual caractere o objeto caiu, e assim, imprimir o caractere $*$ naquele local. O jogador vence quando acertar o adversário, ou seja, quando a distância máxima estiver no intervalo correspondente ao caractere (letra) do adversário.
    
%Quando um objeto for lançado você deve imprimir o local que ele caiu. Você vai representar o objeto com o caractere \textit{0} (zero). A fórmula de Lançamento Inclinado vai retornar a distância máxima de queda do objeto, você vai usar esta distância para descobrir em qual caractere o objeto caiu, e assim,  substituir este caractere pelo caractere \textit{0} (zero). 

Voltando ao exemplo acima, se o jogador $J$ lançar um objeto com uma velocidade inicial de $20 m/s$ e um ângulo de inclinação de $80^{\circ}$, pela fórmula de inclinação, a distância máxima percorrida pelo objeto é de $31.38$ metros, somando com a posição de $J$ ($50$ metros), temos que o objeto caiu a uma distância de $81.38$ metros. Esta distância está no intervalo de $80$ a $90$ metros, que corresponde ao oitavo caractere. Assim na oitavo posição, o código deve imprimir o caractere $*$, indicando a posição onde o objeto caiu, como no exemplo abaixo. 

\begin{center}
    $\_\_\_\_J\_\_*\_\_\_\_\_\_\_\_\_\_\_\_\_\_\_\_\_\_\_\_\_\_\_\_\_\_\_\_\_\_M\_$
\end{center}

Como o jogador $J$ não acertou, passamos a vez para o jogador $M$. Imagine que o jogador $M$ passou os mesmos valores de velocidade e ângulo, note que o lançamento do objeto é sempre direcionado a posição do oponente. O código deve imprimir cenário abaixo.

\begin{center}
    $\_\_\_\_J\_\_\_\_\_\_\_\_\_\_\_\_\_\_\_\_\_\_\_\_\_\_\_\_\_\_\_\_\_\_*\_\_M\_$
\end{center}


Perceba que o objeto lançado por $J$ já desapareceu do cenário. Ou seja, \textbf{apenas o último objeto lançado aparece no cenário.}


\section{Dicas}
\begin{itemize}
\item \textit{isalpha()} é uma função da biblioteca \textit{ctype.h} que verifica se o caractere é uma letra.
\item \textit{sin()} é uma função da biblioteca \textit{math.h} que calcula o seno. Deve ser passado como parâmetro da função o ângulo em radianos ($1^{\circ} = 1 \times \frac{\pi}{180}$ radianos). 
\end{itemize}

\end{document}